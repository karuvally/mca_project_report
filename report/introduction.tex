\chapter{Introduction}

The Netdog project initially started as a simple tool to shutdown computers in a
network. In the course of development, several features were added into it and
it morphed into a easy to use configuration management system.
\\\\
Configuration Management systems are a class of automation software which allows
easy management of computers on a network. Imagine a network where a specific
version of a software needs to be installed across all computers. They can be
achieved through configuration management system.
\\\\
One thing about configuration management systems is that, they are designed for
computer professionals who are not afraid to get their hands dirty and spend
considerable amount of time learning and debugging. This makes these tools not
very accessible to non tech-savvy people. For example, Redhat's Ansible, a
popular configuration management systems expects its users to write the
configuration in YAML format. Not something every literate person in the world
knows.
\\\\
Netdog aims to differ here by offering dead simple interface which can be used
to administer the computers on a network. Amazingly, if a user knows to use the
keyboard and mouse and is aware of the command that is needed to be run across
the machines, he is ready to use Netdog. No further training is needed. But
equally as important, Netdog provides a number of additional features which
makes management of computers on the network a breeze. The following sections
take a more intimate look at Netdog and what it offers.

\section{Configuration}
Netdog provides automated installation and configuration. Make sure the clients
and server can see each other on the network, install the client part on each
client and install the server and you are good to go. This is in stark
difference to normal configuration management systems which require a person to
spend considerable amount of time figuring out how to get it up and running.

\section{Interface}
Netdog uses an easy to use web interface for users to manage the computers on
the network. On the first run of Netdog server, the interface prompts the user
to select the network interface which should be used by the server for
communicating with the clients. The interface is protected by a username and
password which can be chosen by the user at the time of first run.
\\\\
Once logged in, the homepage is presented to the user. From here, Netdog allows
the users to execute commands, Shutdown, Suspend, broadcast files and monitor
all the computers on the network.

\section{Executing commands}
Netdog allows easy execution of commands across all the machines. The commands
are executed on the client machine with super user permission. This means you
can perform system level tasks easily such as installing and updating the
packages on the client as well as make changes to system files.
\\\\
When the user clicks on the Execute command button, he/she is taken first to a
page where the command can be entered. Once that is done, he has to choose the
clients on which the command would get executed. By default, all the clients
are selected. The user is free to deselect the clients by unchecking check
boxes right next to them.

\section{Broadcasting files}
A very useful feature provided by Netdog is the ability to broadcast files
across clients. This too can be done from the home page. Clicking on the
Broadcast button takens the user to the upload page where the source file can
be selected. Once that is done, select the target clients and the file will get
broadcasted to all of them. The broadcasted files will be in
"\textbackslash share" directory in the clients.
\\\\
The interesting thing about broadcast feature is that, it can broadcast any
file, no matter what type or how large. This makes it very convinient as
documents, pictures, videos and even software packages can be broadcasted.

\section{Convenience features}
Netdog provides a couple of convenience features, which allows the users to
shutdown and suspend the clients without having to type the commands for them.
These works very similar to above features. The shutdown and suspend features
have their own buttons in the home page. The user can simply click the button,
select the target machines and the said systems will get affected.
