\chapter{Requirement Analysis}

\section{Purpose}

The purpose is to build a configuration management system that is very simple to
use and at the same time providing enough features so that it is power enough
and can distinguish itself from the rest of the applications in the market. A
second aim was to integrate a monitoring system into the product so that the
system administrators do not have to learn and configure a separate monitoring
application such as Nagios.
\\\\
Though feature rich configuration management systems exist in the market, all of
them are plagued by learning curve. None of them can be used by pointing and
clicking. Netdog aims to fix that and prove that you don't have to be a computer
savvy person to manage computers on a network.

\section{Overall Description}

NetDog uses client-server architecture. The server provides the web interface to
be used by the system administrator. It also provides interfaces for clients to
connect to and exchange information. The server can push information to the
clients and vice versa. Server usually pushes files and execution information
while clients push out status information.
\\\\
After installing NetDog server or client on a machine, a unique public-private
key pair for the machine is generated which is then used for uniquely
identifying the machine and securing data transmission between the client and
server. The advantage of this method is that, even if the IP addresses of the
client or server changes, they can identify each other and re-start functioning
as if nothing happened.
\\\\
Once the server program is up and running, it listens on the port 1337 for
connections from clients. Once the client program is up, it starts listening on
port 1994. These ports serve dual purpose of facilitating communication and
allowing the identification of server and clients from the rest of the machines
on the network. Both NetDog server and client are daemons. They are system 
services which remain in memory and automatically starts during system boot.
They start even before a user logs into the system, ie. Netdog remains fully
functional as long as the computer is powered on.
\\\\
When a client starts for the first time, it looks for active servers on the
network. When it finds one, it starts the pairing procedure. During the pairing
process, the client sends its hostname and public-key. The server in turn
provides the client with it's public key. These keys are then used for
identification and encrypted communication between machines.

\subsection{Product Functions}
\begin{itemize}
    \item Execute commands/scripts remotely on machines
    \item Broadcast files to Netdog clients 
    \item Track and identify clients through IP changes
    \item Secure client-server communication using public key encryption
    \item Monitor all the clients on the network 
    \item Easy to use web interface
\end{itemize}

\subsection{Hardware Requirements}
\begin{itemize}
    \item Intel Pentium IV or equivalent CPU
    \item 512 MB or more RAM
    \item 100 mbps Network Interface Card
\end{itemize}

\subsection{Software Requirements}
\begin{itemize}
    \item Linux
    \item Python 3.5
\end{itemize}

\section{Functional Requirements}
The system should be designed to accept communication requests from many
clients at once. For this, a multi threaded server is necessary. Also, network
outages can occur during the operation. The server should be resilient enough
so that, it checks the status of the connection every once in a while and
restarts the communication process once the network is up and running again.
Also, the functions should extensively log themselves so that in case of a
system failure, the culprit can be easily found.

\section{Performance Requirements}
The system would need a gigabit ethernet controller for the server to make sure
that it can properly handle connections from multiple clients on the network.
The client can work satisfactorily well even on old hardware such as a 10 mbit
network card. There are no specific requirements for the CPU or the rest of the
hardware. The machine must be powerful enough to run a recent version of Linux,
which means any usable computing hardware would do.

